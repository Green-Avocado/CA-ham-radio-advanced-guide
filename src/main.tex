\documentclass[letterpaper]{article}

\usepackage[margin=1in]{geometry}
\usepackage[hidelinks]{hyperref}
\usepackage{amsmath}

\hypersetup{colorlinks,allcolors=blue}

\setcounter{tocdepth}{2}

\begin{document}
    \begin{center}
        \Large
        Canadian Amateur Radio Operator Guide

        Advanced

        \large
        written by VE6SDR

        with notes by VE2AAY
    \end{center}

    \pagenumbering{roman}
    \tableofcontents
    \newpage

    \section*{Keywords}
        \addcontentsline{toc}{section}{Keywords}
        \begin{description}
            \item[farad] The unit of capacitance (symbol: F), 1 farad is the capacitance of a capacitor that has a charge of 1 coulomb when applied voltage drop of 1 volt.
            \item[henry] The unit of inductance (symbol: H), 1 henry is the amount of inductance that causes a voltage of one volt, when the current is changing at a rate of one ampere per second.
            \item[reactance] the nonresistive component of impedance in an AC circuit, arising from the effect of inductance or capacitance or both and causing the current to be out of phase with the electromotive force causing it.
                The unit of reactance is the ohm.
            \item[RC circuit] A resistor–capacitor circuit (RC circuit), or RC filter or RC network, is an electric circuit composed of resistors and capacitors driven by a voltage or current source. A first order RC circuit is composed of one resistor and one capacitor and is the simplest type of RC circuit.
            \item[RL circuit] A resistor–inductor circuit (RL circuit), or RL filter or RL network, is an electric circuit composed of resistors and inductors driven by a voltage or current source. A first-order RL circuit is composed of one resistor and one inductor and is the simplest type of RL circuit.
            \item[RLC circuit] A resistor-inductor-capacitor circuit (RLC circuit) consists of a resistance, a capacitance and an inductance connected to an alternating supply.
            \item[Skin Effect] The tendency of AC to flow in an increasingly thinner layer at the surface of a conductor as frequency increases.
        \end{description}

    \newpage

    \pagenumbering{arabic}
    
    \section{Advanced Theory}
        \subsection{Fundamentals}
        \subsubsection{Fields}
        \begin{description}
            \item[electromagnetic field] the magnetic field created around a conductor carrying current.
            \item[magnetic field] a space around a magnet or a conductor where a magnetic force is present.

                The `Left-hand Rule':
                position the left hand with your thumb pointing in the direction of electron flow;
                encircle the conductor with the remaining fingers, the fingers point in the direction of the magnetic lines of force.
                Using conventional current flow, this would become the Right-hand rule.

                A magnetic field is oriented about a conductor in relation to the direction of electron flow \textbf{in the direction determined by the left-hand rule}.

            \item[electrostatic field] the electric field present between objects with different static electrical charges.

                An electrostatic field is the electric field present between objects with different static electrical charges.
                Voltage across a capacitor creates an electrostatic field between the plates.

            \item[electric field] a space where an electrical charge exerts a force (attraction or repulsion) on other charges.
        \end{description}

        \begin{itemize}
            \item Electromagnetic and electrostatic fields are capable of storing energy. Energy in this state is \textbf{potential energy}.
        \end{itemize}

        \subsubsection{Capacitors}
        Capacitors store energy in an \textbf{electrostatic field}.
        The capacitance in \textbf{farads} is one factor influencing how much energy can be stored in a capacitor.
        One farad accepts a charge of one coulomb when subjected to one volt.

        Capacitive reactance is calculated as follows:
        $$ X_C = \frac{1}{2 \pi f C} $$

        \subsubsection{Inductors}
        Inductors store energy in an electromagnetic field.
        The inductance in \textbf{henries} is one factor influencing how much energy can be stored in an inductor.
        One henry produces one volt of counter EMF with current changing at a rate of one ampere per second.

        Inductive reactance is calculated as follows:
        $$ X_L = 2 \pi f L $$

        \subsubsection{Skin Effect}
        Skin Effect is the tendency of AC to flow in an increasingly thinner layer at the surface of a conductor as frequency increases.

        \begin{itemize}
            \item Skin Effect causes most of an RF current to flow along the surface of a conductor.
            \item The resistance of a conductor different for RF currents than for direct currents because of the Skin Effect.
        \end{itemize}

        \newpage

        \subsection{Resistor-capacitor circuits}
        \subsubsection{Time constant}
        The time constant of an RC circuit (in seconds), is equal to the product of the circuit resistance (in ohms) and the circuit capacitance (in farads).
        $$ \tau = R \cdot C $$
        It is the time required to:
        \begin{itemize}
            \item \textbf{charge} the capacitor from an initial charge voltage of zero to approximately \textbf{63.2\%} of the value of an applied DC \textbf{voltage}, or
            \item \textbf{discharge} the capacitor to approximately \textbf{36.8\%} of its initial charge \textbf{voltage}.
        \end{itemize}

        When charging a capacitor, every time constant reduces the difference to the value of applied DC voltage by 63.2\%.
        $$ \Delta V_f = \Delta V_i \cdot 0.368^{t / \tau} $$
        Where $ \Delta V $ is the difference between the voltage of the capacitor and the applied DC voltage, and $ t $ is time in seconds.

        When discharging a capacitor, every time constant reduces the voltage of the capacitor by 63.2\%.
        $$ V_f = V_i \cdot 0.368^{t / \tau} $$
        Where $ V $ is the voltage of the capacitor and $ t $ is time in seconds.

        \begin{itemize}
            \item Charging a capacitor, the voltage after 1, 2, and 5 times constants is respectively 63\%, 87\%, and 100\% of the final value.
            \item Discharging a capacitor, the voltage is respectively 37\% ($ 100 - 63 $) and 13\% ($ 100 - 87 $) of the initial voltage after 1 and 2 time constants.
        \end{itemize}

        \newpage

        \subsection{Resistor-inductor circuits}
        \subsubsection{Time constant}
        The time constant of an RL circuit (in seconds), is equal to the circuit inductance (in henries) divided by the circuit resistance (in ohms).
        $$ \tau = L / R $$
        It is the time required to:
        \begin{itemize}
            \item \textbf{build} the \textbf{current} in the circuit up to \textbf{63.2\%} of its maximum value.
        \end{itemize}

        When building current in a circuit, every time constant reduces the difference to the maximum value by 63.2\%.
        $$ \Delta I_f = \Delta I_i \cdot 0.368^{t / \tau} $$
        Where $ \Delta I $ is the difference between the current in a circuit and the maximum value, and $ t $ is time in seconds.

        \begin{itemize}
            \item The current after 1, 2 and 5 time constants is respectively 63\%, 87\%, and 100\% of the final value.
        \end{itemize}

        \subsubsection{Counter electromotive force}
        Back EMF or `counter electromotive force' is the voltage induced by changing current in an inductor. It is the force opposing changes in current through inductors.
        \textbf{Back EMF} is \textbf{A voltage that opposes the applied EMF}.

        \newpage

        \subsection{Resistor-inductor-capacitor circuits}
        \subsubsection{Resonance}
        Resonance occurs in an RLC circuit when the supply frequency causes the voltages across L and C to be equal and opposite in phase.
        At resonance, reactances are equal.
        $$ X_L = X_C $$

        The resonant frequency of an RLC circuit is calculated as follows for inductance in henries and capaciance in farads:
        $$ f = \frac{1}{2 \pi \sqrt{LC}} $$

        Restating the equation for frequency in megahertz, inductance in microhenries, and capacitance in picofarads results in the following equation:
        $$ f(\mathrm{MHz}) = \frac{1000}{2 \pi \sqrt{L (\mathrm{\mu H}) \cdot C (\mathrm{p F})}} $$

        These equations can be rearranged to solve for inductance or capacitance as follows:
        \begin{align*}
            L &= \frac{1}{4 \pi^2 f^2 C} \\
            C &= \frac{1}{4 \pi^2 f^2 L}
        \end{align*}

        Or in megahertz, microhenries, and picofarads:
        \begin{align*}
            L (\mathrm{\mu H}) &= \frac{1000000}{4 \pi^2 f (\mathrm{MHz})^2 C (\mathrm{p F})} \\
            C (\mathrm{p F}) &= \frac{1000000}{4 \pi^2 f (\mathrm{MHz})^2 L (\mathrm{\mu H})}
        \end{align*}

        \subsubsection{Quality factor}
        The quality factor or selectivity, $ Q $, is proportional to the inverse of bandwidth for a constant resonant frequency.
        This relationship is illustrated by the fact that resonant frequency is equal to the product of the quality factor and bandwidth of a circuit.
        
        The Q factor can be calculated by dividing resistance by reactance.
        \begin{align*}
            Q &= \frac{R}{X} \\
            Q &= \frac{R}{2 \pi f L} = 2 \pi f C R = R \sqrt{\frac{C}{L}}
        \end{align*}

        A resistor often included in a parallel resonant circuit \textbf{to decrease the Q and increase the bandwidth}.

    \newpage

    \section{Advanced Components and Circuits}
        \subsection{Filters}
        There are 4 categories of filters: high-pass, low-pass, band-pass, and band-stop.

        The three general groupings of filters are \textbf{high-pass, low-pass, and band-pass}.

        Note: \textbf{The bandwidth of a fast-scan TV channel is 6 MHz;  that is much too wide for any of the filters listed}.

        \subsubsection{Crystal filters}
        A crystal lattice filter is \textbf{filter with narrow bandwidth and steep skirts made using quartz crystals}.

        \begin{description}
            \item[Crystal lattice filter] uses two matched pairs of series crystals and a higher-frequency matched pair of shunt crystals in a balanced configuration.
            \item[Half-lattice crystal filter] uses two crystals in an unbalanced configuration.
        \end{description}

        The frequency separation between the crystals (\textbf{relative frequencies of the individual crystals}) sets the bandwidth and the response shape.

        Speech frequencies on a communication-grade SSB voice channel range from 300 hertz to 3000 hertz and thus require a bandwidth of $ 2.7 \mathrm{kHz} $.
        $ 2.4 \mathrm{kHz} $ is a good compromise between fidelity and selectivity.
        \begin{itemize}
            \item For single-sideband phone emissions, $ \mathbf{2.4} \mathrm{\textbf{kHz}} $ would be the bandwidth of a good crystal lattice filter.
        \end{itemize}

        A quartz crystal filter is superior to an LC filter for narrow bandpass applications because of the \textbf{crystal's high Q}.

        Piezoelectric crystals behave like tuned circuits with an extremely high Quality Factor (in excess of 25 000).  Their accuracy and stability are outstanding.
        \begin{itemize}
            \item Electrically, a crystal looks like \textbf{a very high Q tuned circuit}.
        \end{itemize}

        The piezoelectric property of quartz (generating electricity under mechanical stress, bending when subjected to electric field) is used in crystal-based oscillators, radio-frequency crystal filters, such as the lattice filter, and crystal microphones.
        The Active Filter is based on an active device, generally an operational amplifier, and a network of resistors and capacitors.
        \begin{itemize}
            \item Crystal oscillators, filters, and microphones depend upon the \textbf{piezoelectric effect}.
            \item Crystals are not applicable to \textbf{Active Filters}.
        \end{itemize}

        Piezoelectricity is generated by \textbf{deforming certain crystals}.
        The piezoelectric property of quartz is two-fold:
        \begin{itemize}
            \item apply mechanical stress to a crystal and it produces a small electrical field;
            \item subject quartz to an electrical field and the crystal changes dimensions slightly.
        \end{itemize}

        Crystals are capable of resonance either at a fundamental frequency depending on their physical dimensions or at overtone frequencies near odd-integer multiples (3rd, 5th, 7th, etc.).
        Piezoelectric crystals can serve as filters because of their extremely high ``Q'' (greater than 25 000) or as stable, noise-free, and accurate frequency references.

        \subsubsection{Butterworth and Chebyshev filters}
        The Butterworth class of filters exhibit ``maximally flat response'': smooth response, no passband ripple.
        Their frequency response is as flat as mathematically possible in the passband, no bumps or variations (ripple) [first described by British engineer Stephen Butterworth].
        \begin{itemize}
            \item The distinguishing feature of a Butterworth filter is that \textbf{it has a maximally flat response over its pass-band}.
            \item The primary advantage of the Butterworth filter over the Chebyshev filter is that \textbf{it has maximally flat response over its passband}.
        \end{itemize}
        Here is a mnemonic trick: ``The Butterworth's response is smooth as butter''.

        The Chebyshev class of filters [in honour of Pafnuty Chebyshev, a Russian mathematician] have steeper cutoff slopes and more ripple than Butterworth filters.
        Elliptic filters are sharper than the previous two.
        \begin{itemize}
            \item The distinguishing feature of a Chebyshev filter is that \textbf{it allows ripple in the passband in return for steeper skirts}.
            \item The primary advantage of the Chebyshev filter over the Butterworth filter is that \textbf{it allows ripple in the passband in return for steeper skirts}.
            \item \textbf{A Chebyshev filter} is described as having ripple in the passband and a sharp cutoff.
        \end{itemize}

        \subsubsection{Resonant cavities}
        The quarter wavelength Resonant Cavity behaves like a very high Q filter.
        Due to their physical size, they become practical only at VHF frequencies.
        \begin{itemize}
            \item At 50 MHz (6 m), the length of the cavity is \textbf{1.5 m} (one quarter wavelength).
            \item The \textbf{cavity} filter type is not suitable for use at audio and low radio frequencies.
        \end{itemize}

        \subsubsection{Helical resonators}
        The Helical Resonator, based on the concept of a resonant helically-wound section of transmission line within a shielded enclosure, achieves selectivity comparable to the quarter-wave resonant cavity but with a substantial size reduction.
        \begin{itemize}
            \item A device which helps with receiver overload and spurious responses at VHF, UHF and above may be installed in the receiver front end. It is called a \textbf{helical resonator}.
        \end{itemize}

        \newpage

        \subsection{Semiconductors}
        \subsubsection{Materials}
        The most basic semiconductor materials are silicon and germanium.
        Atoms in metallic elements hold their peripheral electrons loosely, such materials make good conductors.
        Peripheral electrons in non-metallic elements are tightly bound, such materials are insulators.
        Germanium and silicon fall somewhere between the two categories but are mostly insulators when pure.
        Doping with impurities increases their conductivity.
        \begin{itemize}
            \item An element which is sometimes an insulator and sometimes a conductor is called a \textbf{semiconductor}.
            \item \textbf{Silicon and germanium} are widely used in semiconductor devices exhibit both metallic and non-metallic characteristics.
            \item Silicon, in its pure form, is \textbf{an insulator}.
            \item A semiconductor is said to be doped when it has added to it small quantities of \textbf{impurities}.
        \end{itemize}

        Pure germanium and silicon are doped with impurities to produce the basic semiconductor materials.
        Certain doping impurities add free electrons, forming N-Type material while others accept electrons, thus creating `holes' found in P-Type material.
        \begin{itemize}
            \item \textbf{P-type} semiconductor material contains fewer free electrons than pure germanium or silicon crystals.
            \item \textbf{Holes} are the majority charge carriers in P-type semiconductor material.
            \item \textbf{N-type} semiconductor material contains more free electrons than pure germanium or silicon crystals.
            \item \textbf{Free electrons} are the majority charge carriers in N-type semiconductor material.
        \end{itemize}

        Gallium arsenide (GaAs) devices can work at higher frequencies with less noise than their silicon counterparts.
        \begin{itemize}
            \item \textbf{At microwave frequencies}, gallium-arsenide used as a semiconductor material in preference to germanium or silicon.
        \end{itemize}

        \subsubsection{Diodes}
        \begin{description}
            \item[Zener diodes] maintain a constant voltage across a range of currents.
            \item[The Varactor] (or Varicap) is a diode used under reverse bias as a ``voltage-variable capacitor''.
            \item[Hot-carrier diodes] (or Schottky-barrier diodes) have lower forward voltage and good high-frequency response:
                \begin{itemize}
                    \item their speed make them useful in Very High Frequency mixers or detectors;
                    \item in power circuits, they are excellent rectifiers in switching power supplies.
                \end{itemize}
            \item[PIN diodes] (with a layer of undoped or lightly doped `intrinsic' silicon between the P and N regions) are used as switches or attenuators.
        \end{description}
        \begin{itemize}
            \item The principal characteristic of a Zener diode is \textbf{a constant voltage under conditions of varying current}.
            \item A Zener diode is a device used to \textbf{regulate voltage}.
            \item A \textbf{Varactor} varies its internal capacitance as the voltage applied to its terminals varies.
            \item A common use for the hot-carrier (Schottky) diode is \textbf{as VHF and UHF mixers and detectors}.
            \item One common use for PIN diodes is \textbf{As an RF switch}.
        \end{itemize}

        Diodes conduct in one direction only:
        under forward bias, maximum forward current is limited by acceptable junction temperature.
        The voltage drop across the junction (volts) multiplied by the forward current (amperes) gives rise to heat dissipation (watts).
        Surviving a reverse bias is determined by the Peak Inverse Voltage (PIV) rating.
        \begin{itemize}
            \item \textbf{Junction temperature} limits the maximum forward current in a junction diode.
            \item \textbf{Maximum forward current and peak inverse voltage (PIV)} are the major ratings for junction diodes.
        \end{itemize}

        There are two main categories of semiconductor diodes:
        \begin{description}
            \item[Point-contact diodes] where a small metal whisker touches the semiconductor material, exhibit low capacitance and serve as RF detectors or UHF mixers.
            \item[Junction diodes] are formed with adjacent blocks of P and N material; these are usable from DC to microwave.
        \end{description}
        \begin{itemize}
            \item A common use for point contact diodes is \textbf{as an RF detector}.
        \end{itemize}

        To calculate power, voltage, or current through a diode:
        $$ P = I V $$
        where $ P $ is power in watts, $ I $ is current in amperes, and $ V $ is voltage in volts.
        \begin{itemize}
            \item For example, if a Zener diode rated at 10 V and 50 W was operated at maximum dissipation rating, it would conduct \textbf{5 A}.
        \end{itemize}

        Heat flows from hot to cold.
        If ambient temperature is higher, less heat can be drained from the junction, the junction will reach maximum safe operating temperature quicker.
        \begin{itemize}
            \item If the temperature is increased, the power handling capability is \textbf{less}.
        \end{itemize}

        \subsubsection{Transistors}

        \newpage

        \subsection{Amplifiers, mixers, and frequency multipliers}

        \newpage

        \section{Measurements}

        \newpage

        \section{Power Supplies}

    \newpage

    \section{Transmitters, Modulation, and Processing}

    \newpage

    \section{Receivers}

    \newpage

    \section{Feedlines - Matching and Antenna Systems}

\end{document}

