\documentclass[letterpaper]{article}

\usepackage[margin=1in]{geometry}
\usepackage[hidelinks]{hyperref}
\usepackage{amsmath}

\hypersetup{colorlinks,allcolors=blue}

\setcounter{tocdepth}{2}

\begin{document}
    \begin{center}
        \Large
        Canadian Amateur Radio Operator Guide

        Advanced

        \large
        written by VE6SDR
    \end{center}

    \pagenumbering{roman}
    \tableofcontents
    \newpage

    \section*{Keywords}
        \addcontentsline{toc}{section}{Keywords}
        \begin{description}
            \item[farad] The unit of capacitance (symbol: F), 1 farad is the capacitance of a capacitor that has a charge of 1 coulomb when applied voltage drop of 1 volt.
            \item[henry] The unit of inductance (symbol: H), 1 henry is the amount of inductance that causes a voltage of one volt, when the current is changing at a rate of one ampere per second.
            \item[reactance] the nonresistive component of impedance in an AC circuit, arising from the effect of inductance or capacitance or both and causing the current to be out of phase with the electromotive force causing it.
                The unit of reactance is the ohm.
            \item[RC circuit] A resistor–capacitor circuit (RC circuit), or RC filter or RC network, is an electric circuit composed of resistors and capacitors driven by a voltage or current source. A first order RC circuit is composed of one resistor and one capacitor and is the simplest type of RC circuit.
            \item[RL circuit] A resistor–inductor circuit (RL circuit), or RL filter or RL network, is an electric circuit composed of resistors and inductors driven by a voltage or current source. A first-order RL circuit is composed of one resistor and one inductor and is the simplest type of RL circuit.
            \item[RLC circuit] A resistor-inductor-capacitor circuit (RLC circuit) consists of a resistance, a capacitance and an inductance connected to an alternating supply.
            \item[Skin Effect] The tendency of AC to flow in an increasingly thinner layer at the surface of a conductor as frequency increases.
        \end{description}

    \newpage

    \pagenumbering{arabic}
    
    \section{Advanced Theory}
        \subsection{Basics}
        \subsubsection{Fields}
        \begin{description}
            \item[electromagnetic field] the magnetic field created around a conductor carrying current.
            \item[magnetic field] a space around a magnet or a conductor where a magnetic force is present.

                The `Left-hand Rule':
                position the left hand with your thumb pointing in the direction of electron flow;
                encircle the conductor with the remaining fingers, the fingers point in the direction of the magnetic lines of force.
                Using conventional current flow, this would become the Right-hand rule.

                A magnetic field is oriented about a conductor in relation to the direction of electron flow \textbf{in the direction determined by the left-hand rule}.

            \item[electrostatic field] the electric field present between objects with different static electrical charges.

                An electrostatic field is the electric field present between objects with different static electrical charges.
                Voltage across a capacitor creates an electrostatic field between the plates.

            \item[electric field] a space where an electrical charge exerts a force (attraction or repulsion) on other charges.
        \end{description}

        \begin{itemize}
            \item Electromagnetic and electrostatic fields are capable of storing energy.

                Energy in this state is \textbf{potential energy}.
        \end{itemize}

        \subsubsection{Capacitors}
        Capacitors store energy in an \textbf{electrostatic field}.
        The capacitance in \textbf{farads} is one factor influencing how much energy can be stored in a capacitor.
        One farad accepts a charge of one coulomb when subjected to one volt.

        Capacitive reactance is calculated as follows:
        $$ X_C = \frac{1}{2 \pi f C} $$

        \subsubsection{Inductors}
        Inductors store energy in an electromagnetic field.
        The inductance in \textbf{henries} is one factor influencing how much energy can be stored in an inductor.
        One henry produces one volt of counter EMF with current changing at a rate of one ampere per second.

        Inductive reactance is calculated as follows:
        $$ X_L = 2 \pi f L $$

        \subsubsection{Skin Effect}
        Skin Effect is the tendency of AC to flow in an increasingly thinner layer at the surface of a conductor as frequency increases.

        \begin{itemize}
            \item Skin Effect causes most of an RF current to flow along the surface of a conductor.
            \item The resistance of a conductor different for RF currents than for direct currents because of the Skin Effect.
        \end{itemize}

        \newpage

        \subsection{Resistor-capacitor circuits}
        \subsubsection{Time constant}
        The time constant of an RC circuit (in seconds), is equal to the product of the circuit resistance (in ohms) and the circuit capacitance (in farads).
        $$ \tau = R \cdot C $$
        It is the time required to:
        \begin{itemize}
            \item \textbf{charge} the capacitor from an initial charge voltage of zero to approximately \textbf{63.2\%} of the value of an applied DC \textbf{voltage}, or
            \item \textbf{discharge} the capacitor to approximately \textbf{36.8\%} of its initial charge \textbf{voltage}.
        \end{itemize}

        When charging a capacitor, every time constant reduces the difference to the value of applied DC voltage by 63.2\%.
        $$ \Delta V_f = \Delta V_i \cdot 0.368^{t / \tau} $$
        Where $ \Delta V $ is the difference between the voltage of the capacitor and the applied DC voltage, and $ t $ is time in seconds.

        When discharging a capacitor, every time constant reduces the voltage of the capacitor by 63.2\%.
        $$ V_f = V_i \cdot 0.368^{t / \tau} $$
        Where $ V $ is the voltage of the capacitor and $ t $ is time in seconds.

        \begin{itemize}
            \item Charging a capacitor, the voltage after 1, 2, and 5 times constants is respectively 63\%, 87\%, and 100\% of the final value.
            \item Discharging a capacitor, the voltage is respectively 37\% ($ 100 - 63 $) and 13\% ($ 100 - 87 $) of the initial voltage after 1 and 2 time constants.
        \end{itemize}

        \newpage

        \subsection{Resistor-inductor circuits}
        \subsubsection{Time constant}
        The time constant of an RL circuit (in seconds), is equal to the circuit inductance (in henries) divided by the circuit resistance (in ohms).
        $$ \tau = L / R $$
        It is the time required to:
        \begin{itemize}
            \item \textbf{build} the \textbf{current} in the circuit up to \textbf{63.2\%} of its maximum value.
        \end{itemize}

        When building current in a circuit, every time constant reduces the difference to the maximum value by 63.2\%.
        $$ \Delta I_f = \Delta I_i \cdot 0.368^{t / \tau} $$
        Where $ \Delta I $ is the difference between the current in a circuit and the maximum value, and $ t $ is time in seconds.

        \begin{itemize}
            \item The current after 1, 2 and 5 time constants is respectively 63\%, 87\%, and 100\% of the final value.
        \end{itemize}

        \subsubsection{Counter electromotive force}
        Back EMF or `counter electromotive force' is the voltage induced by changing current in an inductor. It is the force opposing changes in current through inductors.
        \textbf{Back EMF} is \textbf{A voltage that opposes the applied EMF}.

        \newpage

        \subsection{Resistor-inductor-capacitor circuits}
        \subsubsection{Resonance}
        Resonance occurs in an RLC circuit when the supply frequency causes the voltages across L and C to be equal and opposite in phase.
        At resonance, reactances are equal.
        $$ X_L = X_C $$

        The resonant frequency of an RLC circuit is calculated as follows for inductance in henries and capaciance in farads:
        $$ f = \frac{1}{2 \pi \sqrt{LC}} $$

        Restating the equation for frequency in megahertz, inductance in microhenries, and capacitance in picofarads results in the following equation:
        $$ f(\mathrm{MHz}) = \frac{1000}{2 \pi \sqrt{L (\mathrm{\mu H}) \cdot C (\mathrm{p F})}} $$

        These equations can be rearranged to solve for inductance or capacitance as follows:
        \begin{align*}
            L &= \frac{1}{4 \pi^2 f^2 C} \\
            C &= \frac{1}{4 \pi^2 f^2 L}
        \end{align*}

        Or in megahertz, microhenries, and picofarads:
        \begin{align*}
            L (\mathrm{\mu H}) &= \frac{1000000}{4 \pi^2 f (\mathrm{MHz})^2 C (\mathrm{p F})} \\
            C (\mathrm{p F}) &= \frac{1000000}{4 \pi^2 f (\mathrm{MHz})^2 L (\mathrm{\mu H})}
        \end{align*}

        \subsubsection{Quality factor}
        The quality factor or selectivity, $ Q $, is proportional to the inverse of bandwidth for a constant resonant frequency.
        This relationship is illustrated by the fact that resonant frequency is equal to the product of the quality factor and bandwidth of a circuit.
        
        The Q factor can be calculated by dividing resistance by reactance.
        \begin{align*}
            Q &= \frac{R}{X} \\
            Q &= \frac{R}{2 \pi f L} = 2 \pi f C R = R \sqrt{\frac{C}{L}}
        \end{align*}

        A resistor often included in a parallel resonant circuit \textbf{to decrease the Q and increase the bandwidth}.

    \newpage

    \section{Advanced Components and Circuits}
    \subsection{Crystal lattice filters}

    \newpage

    \section{Measurements}

    \newpage

    \section{Power Supplies}

    \newpage

    \section{Transmitters, Modulation, and Processing}

    \newpage

    \section{Receivers}

    \newpage

    \section{Feedlines - Matching and Antenna Systems}

\end{document}

